\documentclass[fleqn,10pt]{wlscirep}
\usepackage[utf8]{inputenc}
\usepackage[T1]{fontenc}
\newenvironment{Figure}
  {\par\medskip\noindent\minipage{\linewidth}}
  {\endminipage\par\medskip}
\title{Title goes here...} % Article title, use manual lines breaks (\\) to beautify the layout

\author[1, 2*]{Javier Corvillo}
\author[2*]{Verónica Torralba}
\author[2]{Diego Campos}
\author[2*]{Ángel G. Muñoz}
\author[3]{Ana Riviére-Cinnamond}

\affil[1]{Complutense University of Madrid, Department of Earth Science and Astrophysics, Madrid, 28040, Spain}
\affil[2]{Barcelona Supercomputing Center, Earth Sciences Department, 08034, Spain}
\affil[3]{Pan-American Health Organization, Communicable Diseases and Health Analysis, Panama City, 0843-03441, Panama}
\affil[*]{javier.corvillo@bsc.es / veronica.torralba@bsc.es / angel.g.munoz@bsc.es}

%\keywords{Public Health, Vector-borne Diseases, Epidemiology, Climate Change, Climate Services, Environmental Sciences}

\begin{abstract}
This will be the abstract for the paper...
\end{abstract}
\begin{document}

\flushbottom
\maketitle

  \section{Introduction} \label{sec-intro}
  However, the transmission dynamics of these diseases, while known to be partly driven by climate, are not so well understood. Understanding the climatological behavior of these diseases is key to understanding their dynamics and to developing effective strategies for their prevention and control.
  \\
  \\
  Using data from a number of sources, we explore and analyze the behavior of the climate-component of \textit{Aedes}-borne disease transmission, in order to understand its role on the dynamics of disease outbreaks in the context of a changing climate. Our analysis is composed of three different studies: 1) a timescale decomposition of disease transmissibility values, thereby guiding officials to understand the climatology and behaviour of outbreaks for budget and resource allocation; 2) a correlation analysis between transmissibility values and different climate variability indices, such as El Niño Southern Oscillation, in order to understand the effects of natural climate patterns onto Aedes-borne outbreaks; and 3) a causality analysis to solidify findings obtained through correlation, identifying the most relevant predictors and their applicability in a climate-and-health service framework for forecasting the transmissibility of Aedes-borne diseases.

  %------------------------------------------------

  \section{Methods} \label{sec-methods}

  In order to understand the climatological behavior of vector-borne diseases, we first need to understand the behavior of the climate component of the disease transmission. Generally speaking, the basic reproduction number, or $R_0$, is a metric that quantifies the transmissibility of vector-borne diseases, and is defined as the average number of secondary cases generated by a single infected individual in a completely susceptible population. It includes the effects of the vector's biology (whether the vector is present in the area), the human behavior (whether an infected host can transmit the disease by traveling), and the climate (whether the conditions are favorable for the vector to transmit the disease). However, by only integrating the climate component of the disease transmission for the computation of $R_0$, then the metric is more so understood as the role of environmental conditions in the spread of the disease. Thus, this definition of $R_0$, which we can understand as the diseases' environmental suitability, can therefore serve as a first approximation of the role of the climate in the dynamics of vector-borne diseases, and how it can be used to forecast disease outbreaks.
\\
\\
  In this context, our analysis is then composed of three different studies under this definition of $R_0$:

\begin{table*}[t]
  \centering
  \begin{tabular}{l|c|l|c}
  \textbf{Index Name}                 & \multicolumn{1}{l|}{\textbf{Abbreviation}} & \textbf{Periodicity}          & \multicolumn{1}{l|}{\textbf{Pattern Type}} \\ \hline
  Atlantic 3 Index                    & ATL3                                       & Several months to a few years & Oceanic                                    \\
  Indian Ocean Basin                  & IOB                                        & Several months to a few years & Oceanic                                    \\
  Indian Ocean Dipole                 & IOD                                        & Between 2-7 years             & Oceanic                                    \\
  El Niño 3.4 Index                   & Niño 3.4                                   & Between 2-7 years             & Oceanic                                    \\
  North Pacific Meridional Mode       & NPMM                                       & Several months to a few years & Atmospheric                                \\
  South Atlantic Subtropical Dipole 1 & SASD1                                      & Several months to a few years & Oceanic                                    \\
  Southern Indian Ocean Dipole        & SIOD                                       & Several months to a few years & Oceanic                                    \\
  South Pacific Meridional Mode       & SPMM                                       & Several months to a few years & Atmospheric                                \\
  Tropical North Atlantic             & TNA                                        & Several months to a few years & Oceanic                                    
  \end{tabular}%
  }
  \caption{Summary of the climate variability indices used in the analysis used for the correlation and causality studies.}
  \label{tab:climate-variability-indices}
  \end{table*}
    
  \subsection{Analysis 1: Multi-timescale climate decomposition of $R_0$} \label{sec-methods-1-analysis}

  With the intent of isolating the human-driven signal from the natural variability $R_0$ data, a "timescale decomposition" methodology was used to obtain the total variance across different time-scales. 

  \subsubsection{Data} \label{sec-methods-1-data}
  The timescale decomposition analysis was undertaken using $R_0$ outputs from the \textit{\underline{Ae}des} \underline{D}isease \underline{E}nvironmental \underline{S}uitability 2's (AeDES2) monitoring system (citation would go here). The basic reproduction number, or $R_0$, is a metric that quantifies the transmissibility of vector-borne diseases, and is defined as the average number of secondary cases generated by a single infected individual in a completely susceptible population. Though $R_0$ is normally obtained using by combining climate data, vector biology, and human behavior, AeDES2's $R_0$ values consider climate data as the only factor for disease transmission, as the system is designed to be used as a climate service for vector-borne diseases, giving a first approximation of the transmissibility of these diseases. Therefore, while these climate-based $R_0$ values are not strictly a tried-and-true metric for epidemiological purposes, its values are still useful for understanding the climatological behavior of vector-borne diseases in the past, ideal for the purposes of this study.
\\
\\
  The 1980-2021 monthly-mean period of AeDES2's $R_0$ values was selected for the analysis, for a total of 504 months or 167 full seasons. Considering that vector borne diseases are extending to previously unaffected areas due to the effects of man-made climate change, AeDES2's coverage has been increased since its inception to contain global outputs, allowing for a comprehensive analysis of the relationship between climate variability indices and $R_0$ both in current \textit{Aedes} hotspots and emerging regions.

  \subsubsection{Methodology (Time-based)} \label{sec-methods-1-methodology}

  As $R_0$ doesn't follow a clearly defined probability distribution function, the temporal analysis filters a given $R_0$ time-series of any given grid-point by employing a \underline{lo}cally \underline{e}stimated \underline{s}catterplot \underline{s}moothing technique (LOESS). In order to obtain the best smoothing parameter for the analysis, a search was performed over a range of values between 1 and 504 months for the fit of the spatial median of the $R_0$ observational data. The best smoothing parameter for said regression was selected using verification metrics for the goodness of fit of the model: the highest R squared value (RSE), the lowest Akaike Information Criterion (AIC) value, or the lowest GCV value. The GCV value is prioritized over the other two metrics, as it is a more robust measure of the goodness of fit of the model to the data, penalizing overfitting.
  \\
  \\
  Once the ideal smoothing parameter is found for the $R_0$ data, the $R_0$ time-series for each gridpoint is separated into four components: a long-term trend signal (understood to be the trend caused by anthropogenic climate change), an inter-annual signal (year to year), a decadal signal (10-30 years), and lastly, a remainder signal which contains other signals of the data (i.e., inter-annual and inter-decadal variability, among others). Variance maps for each of these four components capture the overall direction of the data over time, as well as the climatological variability of $R_0$ in any given grid-point. 
  \\
  \\
  Once variance maps obtained, Strongest Seasonal Signal Regions (SSSRs) are identified as regions with a significant percentage of variance explained by the seasonal component of the data, to be used in the following parts of the analysis. The boundaries for the selection of SSSR regions are defined by the current Intergovernmental Panel on Climate Change set of reference regions for subcontinental analysis of climate model data (Iturbide et al., 2020).  

  \subsection{Analysis 2: Correlation studies between $R_0$ and climate variability indices} \label{sec-methods-2}

  After analyzing the $R_0$ signal and its variability through timescale decomposition, we assess the impact of several climate variability indices on global $R_0$ values over the chosen 1980-2021 monthly-mean period.  

  \subsubsection{Data} \label{sec-methods-2-data}

  Correlation studies are performed over both global and SSSR regions, using the same $R_0$ data as in the previous analysis. A total of 16 climate variability indices have been used for the correlation analysis. Their respective sources, as well as detrending methods for each, are listed and summarized in Table \ref{tab:climate-variability-indices}.

  \subsubsection{Methodology} \label{sec-methods-2-methodology}

  The correlation analysis was performed using the Pearson correlation coefficient, which quantifies the linear relationship between two variables. The Pearson correlation coefficient is defined as:
  \begin{equation}
      r = \frac{\sum_{i=1}^{n} (x_i - \bar{x})(y_i - \bar{y})}{\sqrt{\sum_{i=1}^{n} (x_i - \bar{x})^2} \sqrt{\sum_{i=1}^{n} (y_i - \bar{y})^2}}
  \end{equation}
  where $x_i$ and $y_i$ are the values of the two variables, $\bar{x}$ and $\bar{y}$ are their respective means, and $n$ is the number of observations. The correlation coefficient ranges from -1 to 1, where -1 indicates a perfect negative correlation, 0 indicates no correlation, and 1 indicates a perfect positive correlation. For computation of statistical significance in correlation, a Monte Carlo method was used, with a p-value threshold of 0.05.

  In order to avoid spurious correlation outputs, anthropogenic signal from the natural variability of the data is isolated by using the long-term trend component obtained in the timescale decomposition analysis. For the climate variability indices, we apply the detrending methods listed in Table \ref{tab:climate-variability-indices}. The correlation analysis is performed over the different seasons over the 1980-2022 period.

  \subsection{Analysis 3: Causality studies between $R_0$ and climate variability indices. Outlining of predictors for disease outbreaks} \label{sec-methods-3}

  Causal-based patterns can be identified after this analysis, which, as opposed to correlation, allow for a more robust foundation for the understanding of the underlying mechanisms between climate variability and $R_0$ patterns. In discarding potentially spurious results obtained through correlation, this causality analysis can be used to outline the most relevant predictors for disease outbreaks. These predictors, in turn, can be used for the refining and building of AeDES2's prediction system for improving the accuracy and skill of the ensemble forecasts respect its predecessor.

  \subsubsection{Data} \label{sec-methods-3-data}

  The datasets that were used for the causality analysis are the same as those employed in assessing the impact of climate variability indices on $R_0$ values across the globe (Section \ref{sec-methods-2-data}).

  \subsubsection{Methodology} \label{sec-methods-3-methodology}

  Causality analysis between $R_0$ and climate variability indices was performed by using Liang-Kleeman's proposed methodology for computing information flow between two entities of a dynamical system (Liang and Kleeman, 2005), quantifying the amount of information that one time series (the climate variability indices) can provide about another time series ($R_0$ patterns). This formalism is based on the concept of transfer entropy, which allows to compute causality as:
  \begin{equation}
      T_{2 \rightarrow 1} = \frac{r}{1-r^2}(r'_{2,\partial 1} - r'_{1,\partial 1})
  \end{equation}
  where $T_{2 \rightarrow 1}$ is the rate of entropy transfer from time series 2 to time series 1, $r$ is the correlation coefficient between the two time series, and $r'_{2,\partial 1}$ and $r'_{1,\partial 1}$ are the partial correlation coefficients of time series 2 and 1 with respect to each other. While normalizing the transfer entropy can help to streamline the analysis, it is not advised for the purposes of this study, as higher correlation values in the denominator of the equation will naturally lead to very high values of transfer entropy that can influence in the normalization process.

  Much like for the correlation analysis, the causality analysis is performed over the different seasons, regions, and time period, with the detrending methods listed in Table \ref{tab:climate-variability-indices}. For the causality analysis, a p-value threshold of 0.05 was used for statistical significance, which, following Liang-Kleeman's causality formalism, has been computed using Fisher's information matrix. 

  %------------------------------------------------

  \section{Results}

  Results text goes here...

  \subsection{Analysis 1: Multi-timescale climate decomposition of $R_0$} \label{sec-results-1}

  \subsection{Analysis 2: Correlation studies between $R_0$ and climate variability indices} \label{sec-results-2} \label{sec-results-2}

  \subsection{Analysis 3: Causality studies between $R_0$ and climate variability indices. Outlining of predictors for disease outbreaks} \label{sec-results-3}

  \section{Discussion}

  Discussion text goes here...

  \section{Conclusion}

  Conclusion text goes here...

% \bibliography{references}

\section*{Figure references}

\section*{Author contributions statement}

Á.M., V.T. and D.C. conceived the methodology to be undertaken in this manuscript. Data sources, code and figures were obtained and developed from the ground up by J.C, who also analysed the results. All authors have reviewed the manuscript. 

\section*{Code and data availability statement}
    Code for the generation of $R_0$ values employed for this study, computed using AeDES2's monitoring system, is available under request, and its values can be visualized in an operational, in-development Shiny App (link) for any region and grid-point. Additionally, the necessary datasets, functions and scripts to generate the maps and plots for this manuscript and supplementary material are available under the following GitHub repository: \url{https://github.com/jacorvillo/monitoring_system_analysis}
\section*{Competing interests}
    The authors declare no competing interests.
\end{document}