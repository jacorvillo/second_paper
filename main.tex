\documentclass[10pt,twocolumn]{wlscirep}
\usepackage[utf8]{inputenc}
\usepackage[T1]{fontenc}
\newenvironment{Figure}
  {\par\medskip\noindent\minipage{\linewidth}}
  {\endminipage\par\medskip}
\title{Can climate patterns predict dengue outbreaks? A causal-based analysis on the role of climate change in \textit{Aedes}-borne disease transmission}

\author[1, 2*]{Javier Corvillo}
\author[2*]{Verónica Torralba}
\author[2]{Diego Campos}
\author[3]{Ana Riviére-Cinnamond}
\author{Ángel G. Muñoz}

\affil[1]{Complutense University of Madrid, Department of Earth Science and Astrophysics, Madrid, 28040, Spain}
\affil[2]{Barcelona Supercomputing Center, Earth Sciences Department, 08034, Spain}
\affil[3]{Pan-American Health Organization, Communicable Diseases and Health Analysis, Panama City, 0843-03441, Panama}
\affil[*]{javier.corvillo@bsc.es / veronica.torralba@bsc.es / angel.g.munoz@bsc.es}


\begin{abstract}
  \textbf{Background}: Vector-borne diseases transmitted by \textit{Aedes} mosquitoes such as dengue, Zika, and chikungunya, pose significant public health challenges worldwide in the wake of human-driven climate change. However, while their transmission is known to be susceptible to some climate variables like temperature or the amount of rainfall, the overall role of climate patterns on the emergence of these diseases is not so well understood.
  \\
  \\
\textbf{Methods}: Using data from a number of sources, we explore and analyze the response of \textit{Aedes}-borne disease transmission to climate patterns, in order to understand its influence on disease outbreaks. Our analysis is composed of three different studies: 1) a timescale decomposition of disease transmissibility values, thereby guiding officials to understand the behavior of outbreaks for budget and resource allocation; 2) a correlation analysis between transmissibility values and different climate patterns, such as El Niño Southern Oscillation, in order to understand the effects of natural climate patterns onto \textit{Aedes}-borne outbreaks; and 3) a causality analysis to solidify findings obtained through correlation, identifying the most relevant predictors and their applicability in a climate-and-health service framework for forecasting the transmissibility of \textit{Aedes}-borne diseases.
\\
\\
\textbf{Results}: 
\\
\\
\textbf{Conclusions}: 
\\
\\
\textbf{Keywords}: Public Health, Vector-borne Diseases, Epidemiology, Climate Change, Climate Services, Environmental Sciences
\end{abstract}
\twocolumn
\begin{document}

\flushbottom
\maketitle

\section{Multilingual abstracts} \label{sec-abstract}

Please consult the \hyperref[sec-additional-files]{Additional files} section for abstract translations into the other five official languages of the United Nations (Arabic, Chinese, French, Russian and Spanish).

\section{Background} \label{sec-background}

\subsection{The emergence of vector-borne diseases in the context of climate change} \label{sec-background-vector-borne-diseases}

Disease stability and transmissibility under changing climate conditions has long been a topic of interest and research in the fields of epidemiology and virology. Many viral, bacterial, and parasitic diseases have been shown to be susceptible to changes in environmental conditions across different regions and timescales \cite{thomson_2008, malloy_2019}. This is particularly true for previously pandemic diseases that have become endemic once proper disease control mechanisms are implemented by public health officials \cite{li_2019} . Prevalent respiratory viruses such as H1N1 influenza or the novel SARS-CoV-2 virus have been shown to reduce their dependency on human spread once losing their pandemic status, adopting defined climate-dependent seasonality patterns and becoming more prominent in temperate climates during the winter season \cite{shaman_2011, romerostarke_2021}.
\\
\\
For vector-borne diseases (VBDs), the relationship between climate and pathogen transmissibility is even more intertwined. Carriers such as arthropods, snails and slugs thrive under specific climate-dependent thresholds and conditions, and in the context of anthropogenic climate change, the effects of global warming and changing precipitation patterns have been shown to affect the distribution and behavior of these vectors\cite{lowe_2018, messina_2016}.
\\
\\
Mosquitoes of the \textit{Aedes} genus, such as \textit{Aedes albopictus} and \textit{Aedes aegypti}, are of particular interest and importance in medium and low income countries \cite{campbell-lendrum_2015}. As the main carriers of diseases like dengue (DENV), chikungunya (CHIKV), and Zika (ZIKV), they pose a significant public health threat, traditionally in tropical and subtropical regions\cite{OMS_2020}. In the Americas, for instance, DENV impacts account for over 2 million disability adjusted life years worldwide \cite{yang_2021}, and a low estimated annual cost of about US\$2.1 billion \cite{shepard_2011}.
\\
\\
However, the effects of climate change are causing Aedes-related diseases not only to emerge in new, previously unaffected regions, but also to increase their spread in areas where they were previously endemic\cite{quam_2015}. Along with compounded changes in urbanization\cite{lee_2021a} and population growth\cite{struchiner_2015}, climate change is believed to be a major driver of increased DENV incidence in temperate climates\cite{kraemer_2015}, with the recent establishment of epidemic activity in parts of North America\cite{franklinos_2019} or Southeast Asia\cite{ooi_2009}, and detection of local transmission in southern European countries along the Mediterranean Basin\cite{ECDC_2024c}. Equatorial tropical and subtropical zones like the Sub-Saharan Africa, Southeast Asia, and northern South America have also been subject to higher incidence over the past 40 years\cite{nakase_2024}. While the current yearly incidence of DENV amounts to an average of 400 million cases per year\cite{pourzangiabadi_2025}, it is believed that the effects of climate change may put an additional 2.5 billion people at risk of DENV if Aedes vectors were present in every region where the climate is suitable for their development\cite{nakase_2024}.


\subsection{Known drivers on \textit{Aedes}-borne disease transmission dynamics and their impact} \label{sec-background-aedes-borne-diseases}

The current scientific literature highlights a profound influence of the environmental conditions over \textit{Aedes} mosquito proliferation, with temperature and precipitation being the two primary climate drivers of Aedes-borne disease transmission due to their impact on both vector and pathogen biology alike.
\\
\\
Many vector physiological processes (e.g., biting frequency, reproduction rates) as well as pathogen characteristics (e.g., extrinsic incubation duration) are conditioned by temperature, increasing until certain temperature thresholds are reached\cite{mordecai_2019}. In a similar manner, rainfall promotes the availability of mosquito breeding sites up until a critical point, in which excessive rainfall may flooding or washing out the mosquito larvae away\cite{paaijmans_2007}. In conjunction, the combined effects of temperature and rainfall play a crucial role in the length of the Aedes-borne transmission season of DENV, CHIKV and ZIKV in temperate areas and affected hotspots, with higher incidence particularly in urban areas of the Western Pacific and the Eastern Mediterranean regions\cite{colon-gonzalez_2021}.
\\
\\
Aside from these climate variables, humidity is also another probable factor in the proliferation of \textit{Aedes} mosquito breeding sites, though proper characterization and relationships with disease transmission remains an object of study. Current evidence suggests that elevated humidity levels could play a role in reducing incubation periods and blood-feeding cycle duration in Aedes mosquitoes\cite{descloux_2012}.

\subsection{Climate patterns as an aggregate of unknown transmissibility drivers} \label{sec-climate-patterns}

While the aforementioned macro climatic factors have been widely used in \textit{Aedes}-borne disease research and modelling\cite{caminade_2017, liu-helmersson_2014, mordecai_2017, wesolowski_2015}, it is widely acknowledged that these are not enough to fully explain the climate component of \textit{Aedes}-borne disease transmissibility, and that more explanatory variables can aid in building more refined and actionable DENV monitoring and prediction systems\cite{erraguntla_2021, lee_2017, yavarinejad_2021, sriklin_2021}. For instance, temperature and rainfall my also affect other possible small-scale climate modulators like soil moisture or vegetation growth, which in turn may affect the availability of breeding sites for \textit{Aedes} mosquitoes. Additionally, analysis of high resolution satellite imagery has suggested that land cover change, through deforestation, as well as stagnant water bodies, may also play a role in the proliferation of \textit{Aedes} breeding sites\cite{ali_2025}.
\\
\\
From a pragmatic perspective, understanding the contribution of each of these additional climate variables, each operating in different spatial and temporal scales, can be a challenging undertaking. Moreover, the effects of these variables may not be immediately apparent, as variables like soil moisture, for instance, do depend on many sub-processes.
\\
\\
It is for this reason why climate patterns, in this context, may serve a better role in understanding the underlying climate processes behind \textit{Aedes}-borne disease transmission. Climate patterns are large-scale recurring ocean or atmospheric phenomena that can influence weather and climate conditions over different timescales (subseasonal, seasonal, inter-annual, decadal, etc.). They are often characterized by their periodicity, identified through their defined climate variability indices, and their effects ripple through the climate system, affecting temperature, precipitation, and other climate variables across different regions through teleconnections\cite{lubbecke_2018}.
\\
\\
As such, climate patterns may serve as aggregators of both large and small climate phenomena by linking various climatic oscillations to regional weather events and other environmental processes\cite{lee_2018}. The analysis of known seasonal patterns such as the El Niño Southern Oscillation (ENSO) could provide a framework for understanding complex interactions that influence precipitation, temperature, and ultimately, the role of these compound climate variables processes on \textit{Aedes}-disease transmission. Moreover, since they encapsulate broader climatic trends that affect local conditions, climate patterns seem preferable over everyday climate variables in the development of future climate-and-health \textit{Aedes} prediction systems\cite{easterbrook_2016, hallett_2004}.

\subsection{The applicability of climate services in vector-borne disease prevention} \label{sec-climate-services}

\section{Methods} \label{sec-methods}

We utilize a select number of datasets and methodologies to undertake three distinct analyses, in order to characterise the behavior of \textit{Aedes}-borne disease transmissibility, and to understand the role of climate patterns in their potential predictibility.
\\
\\
By using global climate products that transform climate variables into the climate-driven component of \textit{Aedes}-borne transmissibility, we can explore the behavior of vector-borne diseases at multiple timescales (seasonal, inter-annual, decadal, and long-term trends). We later employ a series of correlation and causality analyses in order to understand the role of climate patterns on disease emergence across regions and seasons. By highlighting which climate patterns are dominant over the different areas and seasons, we can assess their impact on the transmissibility of \textit{Aedes}-borne diseases, therefore serving as predictors for disease outbreaks.

\subsection{Redefining $R_0$ as a bridge between climate and health} \label{sec-methods-redefining-R0}

In order to understand the climatological behavior of \textit{Aedes}-borne diseases, we first need to understand the behavior of the climate component of the disease transmission. Generally speaking, the basic reproduction number, or $R_0$, is a commonly used epidemiological metric that quantifies the transmissibility of vector-borne diseases, and is defined as the average number of secondary cases generated by a single infected individual in a susceptible population. It includes the effects of the vector's behavior (e.g. absence or presence in the area), human behavior (e.g whether an infected host can transmit the disease by traveling), and the climate (whether the conditions are favorable for the vector to transmit the disease). However, by only integrating the climate component of the disease transmission for the computation of $R_0$, then the resulting $R_0$ metric is more so interpreted as the role of environmental conditions in the spread of the disease. Thus, this definition of $R_0$, which we can understand as the diseases' environmental suitability, can be understood as a first approximation of the role of the climate in the dynamics of vector-borne diseases.
\\
\\
In this context, our analysis is then composed of three different studies under this definition of $R_0$:

\subsection{The Aedes Disease Environmental Suitability 2's monitoring system} \label{sec-methods-aedes2}

The \textit{\underline{Ae}des} \underline{D}isease \underline{E}nvironmental \underline{S}uitability 2's (AeDES2) monitoring system was used throughout this study in order to obtain the $R_0$ values for the analysis. Improving over its predecessor, AeDES2 is a climate-and-health service that provides real-time monitoring of the environmental suitability for \textit{Aedes}-borne diseases. The system uses a climate variables like temperature and precipitation, and through its integration with ento-epidemiological models, as well as with calibration with recorded DENV cases, $R_0$ values are computed for different regions and seasons. The monitoring system is designed to be used by public health officials and researchers to assess the risk of disease outbreaks and to inform prevention and control strategies.

\subsection{Analysis 1: Multi-timescale climate decomposition of $R_0$} \label{sec-methods-1-analysis}

\begin{table*}[t]
  \centering
  \begin{tabular}{l|c|l|c}
    \textbf{Index Name}               & \multicolumn{1}{l|}{\textbf{Abbreviation}} & \textbf{Periodicity}          & \multicolumn{1}{l|}{\textbf{Pattern Type}} \\ \hline
    Atlantic 3 Index                  & ATL3                                       & Several months to a few years & Oceanic                                    \\
    Indian Ocean Basin                & IOB                                        & Several months to a few years & Oceanic                                    \\
    Indian Ocean Dipole               & IOD                                        & Between 2-7 years             & Oceanic                                    \\
    El Niño 3.4 Index                 & Niño 3.4                                   & Between 2-7 years             & Oceanic                                    \\
    North Pacific Meridional Mode     & NPMM                                       & Several months to a few years & Atmospheric                                \\
    South Atlantic Subtropical Dipole & SASD                                       & Several months to a few years & Oceanic                                    \\
    Southern Indian Ocean Dipole      & SIOD                                       & Several months to a few years & Oceanic                                    \\
    South Pacific Meridional Mode     & SPMM                                       & Several months to a few years & Atmospheric                                \\
    Tropical North Atlantic           & TNA                                        & Several months to a few years & Oceanic
  \end{tabular}%
  \caption{Summary of the climate variability indices used in the analysis used for the correlation and causality studies.}
  \label{tab:climate-variability-indices}
\end{table*}
With the intent of isolating the human-driven signal from the natural variability $R_0$ data, a timescale decomposition methodology was used to obtain the total variance across different time-scales. This approach allows us to separate the complex $R_0$ signal into distinct temporal components, each providing insight into the underlying climatology of \textit{Aedes}-borne disease transmission. This decomposition is particularly crucial in regions endemic to \textit{Aedes}-borne diseases, such as tropical and subtropical areas of Latin America, Southeast Asia, and parts of Africa, where understanding how different climate processes operate at various temporal scales to influence disease transmissibility is essential for developing effective public health strategies. In these hotspot regions, the ability to distinguish between predictable seasonal patterns, multi-year climate oscillations, and long-term warming trends enables health authorities to optimize resource allocation, implement targeted intervention strategies, and develop more accurate early warning systems for disease outbreak prevention.

\subsubsection{Data} \label{sec-methods-1-data}
The timescale decomposition analysis was undertaken using $R_0$ outputs from the AeDES2's monitoring system. The 1980-2021 monthly-mean period of AeDES2's $R_0$ values was selected for the analysis, for a total of 504 months or 167 full seasons. Considering that vector borne diseases are extending to previously unaffected areas due to the effects of man-made climate change, AeDES2's coverage has been increased since its inception to contain global outputs, allowing for a comprehensive analysis of the relationship between climate variability indices and $R_0$ both in current \textit{Aedes} hotspots and emerging regions.

\subsubsection{Methodology} \label{sec-methods-1-methodology}

As $R_0$ doesn't follow a clearly defined probability distribution function, the temporal analysis filters a given $R_0$ time-series of any given grid-point by employing the non-parametric \underline{lo}cally \underline{e}stimated \underline{s}catterplot \underline{s}moothing technique (LOESS). Sensitivity tests have been conducted in order to obtain the best LOESS smoothing parameter for the analysis, using three verification metrics for the goodness of fit of the model: the highest R squared value ($R^2$), the lowest Akaike Information Criterion ($AIC$) value, or the lowest Generalized Correlated Cross-Validation ($GCV$) value. Whenever these metrics yield conflicting results, the GCV value is prioritized as the primary selection criterion. Unlike $R^2$, which can artificially inflate with increased model complexity, or $AIC$, which relies on asymptotic assumptions that may not hold for finite samples, GCV provides a more robust assessment of model generalizability by directly penalizing overfitting through its leave-one-out validation procedure.
\\
\\
Once the ideal LOESS smoothing parameter is found for the $R_0$ data, the $R_0$ time-series for each grid-point is separated into four components: a long-term trend signal (understood to be the trend caused by anthropogenic climate change), an inter-annual signal (year to year), a decadal signal (10-30 years), and lastly, a remainder signal which contains other signals of the data (i.e., inter-annual and inter-decadal variability, among others). Variance maps for each of these four components capture the overall direction of the data over time, as well as the climatological variability of $R_0$ in any given grid-point.
\\
\\
Variance maps, as well as any results from following analyses, are shown for both global outputs and the Panama region. The Panama region is selected as a case study for the analysis, as it is known to be a present hotspot for \textit{Aedes}-borne diseases, with a long history of DENV outbreaks and a complex interplay of climatic factors that influence disease transmission.
\\
\\
After the variance maps are obtained, $R_0$ values are then detrended for the following correlation and causality analyses. While detrending through the assumption that $R_0$ changes linearly over time could be a valid approach, it fails to capture the temperature dependency of the data, expressed extensively in the literature. In order to capture this temperature dependency, a similar timescale decomposition analysis is performed on the detrended $R_0$ data, but using temperature data from the AeDES2's monitoring system datasets as the independent variable (monthly-mean temperature data consisting of the GHCN-CAMS project, the CPC Unified Global Temperature dataset, the ERA5 reanalysis dataset, and the ERA5Land reanalysis dataset). In this way, the obtained trend serves as a functional relationship between temperature and $R_0$: a temperature-based description of the $R_0$ signal, attributed to the warming of the planet.

\subsection{Analysis 2: Correlation studies between $R_0$ and climate variability indices} \label{sec-methods-2}

After analyzing the $R_0$ signal and its variability through timescale decomposition, we assess the impact of several climate variability indices on global $R_0$ values over the chosen 1980-2021 monthly-mean period.

\subsubsection{Data} \label{sec-methods-2-data}

Correlation studies are performed over both global and Panama regions, using the temperature-based detrended $R_0$ data as in the previous analysis over the different seasons. A total of 9 temperature-based climate variability indices have been used for the correlation analysis, which have been computed using the detrended temperature data utilized in the previous analysis. Their periodicity, as well as their main pattern type, are listed and summarized in Table \ref{tab:climate-variability-indices}.


\subsubsection{Methodology} \label{sec-methods-2-methodology}

The correlation analysis was performed using the Pearson correlation coefficient, which quantifies the linear relationship between two variables. For computation of statistical significance in correlation, the non-parametric Monte Carlo method was used, with a p-value threshold of 0.05.

\subsection{Analysis 3: Causality studies between $R_0$ and climate variability indices. Outlining of predictors for disease outbreaks} \label{sec-methods-3}

Causal-based patterns can be identified after this analysis, which allow for a more robust foundation for the understanding of the underlying mechanisms between climate variability and $R_0$ patterns. These causality studies are performed over both global and Panama regions and over the different seasons. In discarding potentially spurious results obtained through correlation, this causality analysis can be used to outline the most relevant predictors for disease outbreaks. These predictors, in turn, can be used for the refining and building of AeDES2's prediction system for improving the accuracy and skill of the ensemble forecasts compared to its predecessor.

\subsubsection{Data} \label{sec-methods-3-data}

The datasets that were used for the causality analysis are the same detrended datasets as those employed in Section \ref{sec-methods-2-data}.

\subsubsection{Methodology} \label{sec-methods-3-methodology}

Causality analysis between $R_0$ and climate variability indices was performed by using Liang-Kleeman's proposed methodology for computing information flow between two entities of a dynamical system, quantifying the amount of information that one time series (the climate variability indices) can provide about another time series ($R_0$ patterns). Once the transfer entropy is computed, it is then normalized in order to account for the different scales of the two time series. Statistical significance is computed using Fisher's information matrix, with a p-value threshold of 0.05.

\section{Results}

The most relevant results from each of the three analyses are summarized over the following sections of the manuscript, highlighting the most relevant findings and their implications. The complete results from the analyses described above, including the correlation and causality study maps for each individual climate pattern over the different seasons, can all be found in the supplementary material provided along with this manuscript (see \hyperref[sec-additional-files]{Additional files}).


\subsection{Analysis 1: Multi-timescale climate decomposition of $R_0$} \label{sec-results-1}



\subsection{Analysis 2: Correlation studies between $R_0$ and climate variability indices} \label{sec-results-2} \label{sec-results-2}

\subsection{Analysis 3: Causality studies between $R_0$ and climate variability indices. Outlining of predictors for disease outbreaks} \label{sec-results-3}

\section{Discussion} \label{sec-discussion}

\subsection{The added value of climate patterns in seasonal forecasting of \textit{Aedes}-borne diseases} \label{sec-discussion-added-value}

\subsection{Analysis 1: Multi-timescale climate decomposition of $R_0$} \label{sec-discussion-analysis-1}

\subsection{Analyses 2 and 3: Correlation and causality studies between $R_0$ and climate variability indices} \label{sec-discussion-analysis-2-3}

\subsection{Notable limitations and constraints} \label{sec-discussion-limitations}

\section{Conclusions} \label{sec-conclusions}

\begin{enumerate}
  \item Historical and climatological analyses of $R_0$ values for \textit{Aedes}-borne diseases provide insight in understanding the role of climate in disease emergence, and can be used to improve the accuracy of seasonal forecasts through the identification of climate predictors. However, while global climate models are suitable for a broad, general-purpose understanding, high-resolution data is preferred when more nuanced analysis are performed in endemic regions, in order to provide more accurate and actionable information for public health officials.
  \item
\end{enumerate}

\section{Additional files} \label{sec-additional-files}

\section{Abbreviations} \label{sec-abbreviations}
\begin{itemize}
  \item AeDES2: \textit{Ae}des \textit{D}isease \textit{E}nvironmental \textit{S}uitability 2
  \item AIC: Akaike Information Criterion
  \item ATL3: Atlantic 3 Index
  \item CHIKV: Chikungunya
  \item DENV: Dengue
  \item ENSO: El Niño Southern Oscillation
  \item GCV: Generalized cross-validation
  \item IOB: Indian Ocean Basin
  \item IOD: Indian Ocean Dipole
  \item LOESS: Locally estimated scatterplot smoothing
  \item NPMM: North Pacific Meridional Mode
  \item $R^2$: Coefficient of determination
  \item SASD: South Atlantic Subtropical Dipole
  \item SIOD: Southern Indian Ocean Dipole
  \item SPMM: South Pacific Meridional Mode
  \item TNA: Tropical North Atlantic
  \item VBDs: Vector-borne diseases
  \item ZIKV: Zika
\end{itemize}

\section{Acknowledgements} \label{sec-acknowledgements}

\section{Funding} \label{sec-funding}

\section{Availability of data and materials} \label{sec-availability}
Code for the generation of $R_0$ values employed for this study, computed using AeDES2's monitoring system, is available under request, and its values can be visualized in an operational, in-development Shiny App (link) for any region and grid-point. Additionally, the necessary datasets, functions and scripts to generate the maps and plots for this manuscript and supplementary material are available under the following GitHub repository: \url{https://github.com/jacorvillo/monitoring_system_analysis}

\section{Authors' contributions} \label{sec-authors-contributions}

Á.M., V.T. and D.C. conceived the methodology to be undertaken in this manuscript. Data sources, code and figures were obtained and developed from the ground up by J.C, who also analysed the results. All authors have reviewed the manuscript.

\section{Ethics approval and consent to participate} \label{sec-ethics-approval}
Not applicable.

\section{Consent for publication} \label{sec-consent-for-publication}
Not applicable.

\section{Competing interests} \label{sec-competing-interests}
The authors declare no competing interests.

\section{Author details} \label{sec-author-details}

\section{References} \label{sec-references}

\bibliography{references}

\end{document}