\documentclass[10pt]{article}
\usepackage[utf8]{inputenc}
\usepackage[T2A]{fontenc}

\usepackage[
	top=2.5cm, % Top margin
	bottom=2.5cm, % Bottom margin
	left=3cm, % Left margin
	right=3cm, % Right margin
	footskip=1cm, % Space from the bottom margin to the baseline of the footer
	headsep=0.75cm, % Space from the top margin to the baseline of the header
	columnsep=20pt, % Space between text columns (in twocolumn mode)
	%showframe % Uncomment to show frames around the margins for debugging purposes
]{geometry}

% Set left and right margins
\setlength{\rightmargin}{0.25in}
\setlength{\leftmargin}{0.25in}


%\usepackage[LAE]{fontenc} %Not needed due to [arabic] option of the babel package

%Hyphenation rules
%--------------------------------------
\usepackage{hyphenat}
\hyphenation{ма-те-ма-ти-ка вос-ста-нав-ли-вать}
%--------------------------------------
\usepackage[english, russian]{babel}

% Font settings for different scripts

\title{Multilingual Abstracts\\ \\
Can ENSO predict dengue outbreaks? A causal-based analysis on the role of climate patterns in \textit{Aedes}-borne disease transmission}

\author{Javier Corvillo et al.}

\begin{document}

\section*{Abstract Translations}

This document contains translations of the research abstract into the five other official languages of the United Nations, as referenced in the main paper.

\subsection*{English (Original)}

\textbf{Background}: Vector-borne diseases transmitted by \textit{Aedes} mosquitoes such as dengue, Zika, and chikungunya, pose significant public health challenges worldwide in the wake of human-driven climate change. However, while their transmission is known to be susceptible to some climate variables like temperature, rainfall or humidity, the overall role of climate patterns on the emergence of these diseases is not so well understood.
\\
\\
\textbf{Methods}: Using data from a number of sources, we explore and analyse the response of \textit{Aedes}-borne disease transmission to climate patterns, in order to understand its influence on disease outbreaks. Our analysis is composed of three different studies: 1) a timescale decomposition of disease transmissibility values, thereby guiding officials to understand the behaviour of outbreaks for budget and resource allocation; 2) a correlation analysis between transmissibility values and different climate patterns, such as El Niño Southern Oscillation, in order to understand the effects of natural climate patterns onto \textit{Aedes}-borne outbreaks; and 3) a causality analysis to solidify findings obtained through correlation, identifying the most relevant predictors for \textit{Aedes}-borne diseases.
\\
\\
\textbf{Results}: Long-term, man-made climate change is shown to have a significant impact on the environmental suitability for \textit{Aedes}-borne diseases in the tropics, where El Niño Southern Oscillation and the Indian Ocean Basin are key climate patterns conditioning disease emergence. Temperate regions are shown to be more susceptible to seasonal climate variability, where multiscale climate patterns, through teleconnections and compound interactions, can influence transmission dynamics.
\\
\\
\textbf{Conclusions}: Disease transmission of \textit{Aedes}-borne diseases is shown to be susceptible to multiple factors, including long-term climate change and seasonal variability. The results of this study highlight the use of understanding the multi-faceted role of climate patterns in disease emergence, and their potential applicability in the development of early warning systems for \textit{Aedes}-borne disease outbreaks. Future research should focus on integrating these findings into an actionable \textit{Aedes}-borne seasonal prediction system using climate patterns as predictors, which can ultimately better inform public health strategies for future dengue outbreaks.
\\
\\
\textbf{Keywords}: Public Health, Vector-borne Diseases, Epidemiology, Climate Change, Climate Services, Environmental Sciences

\newpage

\subsection*{Español (Spanish)}

\textbf{Contexto}: Las enfermedades vectoriales transmitidas mosquitos de la familia \textit{Aedes} como el dengue, Zika y chikungunya, representan desafíos significativos de salud pública a nivel mundial en el actual contexto de cambio climático antropogénico. Sin embargo, aunque se sabe que su transmisión es susceptible a algunas variables climáticas como la temperatura, las precipitaciones o la humedad, el papel general de los patrones climáticos en la emergencia de estas enfermedades no está bien comprendido.
\\
\\
\textbf{Métodos}: Utilizando datos de diversas fuentes, exploramos y analizamos la respuesta de la transmisión de enfermedades transmitidas por \textit{Aedes} a los patrones climáticos, con el fin de entender su influencia en los brotes de enfermedades. Nuestro análisis está compuesto por tres estudios diferentes: 1) una descomposición de escalas temporales de los valores de transmisibilidad de enfermedades, guiando así a las autoridades sanitarias a entender el comportamiento de los brotes para la asignación de presupuesto y recursos; 2) un análisis de correlación entre los valores de transmisibilidad y diferentes patrones climáticos, como la Oscilación del Sur de El Niño, para entender los efectos de los patrones climáticos en los brotes transmitidos por \textit{Aedes}; y 3) un análisis de causalidad para solidificar los hallazgos obtenidos a través de la correlación, identificando los predictores más relevantes para las enfermedades transmitidas por \textit{Aedes}.
\\
\\
\textbf{Resultados}: Se demuestra que el cambio climático antropogénico a largo plazo tiene un impacto significativo en la idoneidad ambiental para las enfermedades transmitidas por \textit{Aedes} en los trópicos, donde la Oscilación del Sur de El Niño y la Cuenca del Océano Índico son patrones climáticos clave que condicionan la emergencia de enfermedades. Las regiones templadas se muestran más susceptibles a la variabilidad climática estacional, donde los patrones climáticos, a través de teleconexiones e interacciones compuestas, pueden influir en la dinámica de transmisión.
\\
\\
\textbf{Conclusiones}: Se demuestra que la transmisión de enfermedades transmitidas por \textit{Aedes} es susceptible a múltiples factores, incluyendo el cambio climático a largo plazo y la variabilidad estacional. Los resultados de este estudio destacan el uso de comprender el papel multifacético de los patrones climáticos en la emergencia de enfermedades, y su aplicabilidad potencial en el desarrollo de sistemas de alerta temprana para brotes de enfermedades transmitidas por \textit{Aedes}. La investigación futura debe enfocarse en integrar estos hallazgos en un sistema de predicción estacional accionable para enfermedades transmitidas por \textit{Aedes} usando patrones climáticos como predictores, que puede informar mejor las estrategias de salud pública para futuros brotes de dengue.
\\
\\
\textbf{Palabras clave}: Salud Pública, Enfermedades Transmitidas por Vectores, Epidemiología, Cambio Climático, Servicios Climáticos, Ciencias Ambientales

\newpage

\subsection*{Français (French)}

\textbf{Contexte}: Les maladies à transmission vectorielle transmises par les moustiques \textit{Aedes} telles que la dengue, le Zika et le chikungunya, posent des défis significatifs de santé publique dans le monde entier dans le sillage du changement climatique d'origine humaine. Cependant, bien que leur transmission soit connue pour être sensible à certaines variables climatiques comme la température, les précipitations ou l'humidité, le rôle global des modèles climatiques sur l'émergence de ces maladies n'est pas si bien compris.
\\
\\
\textbf{Méthodes}: En utilisant des données de diverses sources, nous explorons et analysons la réponse de la transmission des maladies transmises par \textit{Aedes} aux modèles climatiques, afin de comprendre son influence sur les épidémies de maladies. Notre analyse est composée de trois études différentes : 1) une décomposition d'échelles temporelles des valeurs de transmissibilité des maladies, guidant ainsi les officiels pour comprendre le comportement des épidémies pour l'allocation du budget et des ressources ; 2) une analyse de corrélation entre les valeurs de transmissibilité et différents modèles climatiques, tels que l'Oscillation Australe d'El Niño, afin de comprendre les effets des modèles climatiques naturels sur les épidémies transmises par \textit{Aedes} ; et 3) une analyse de causalité pour solidifier les résultats obtenus par corrélation, identifiant les prédicteurs les plus pertinents pour les maladies transmises par \textit{Aedes}.
\\
\\
\textbf{Résultats}: Le changement climatique anthropique à long terme est montré avoir un impact significatif sur la pertinence environnementale pour les maladies transmises par \textit{Aedes} dans les tropiques, où l'Oscillation Australe d'El Niño et le Bassin de l'Océan Indien sont des modèles climatiques clés conditionnant l'émergence des maladies. Les régions tempérées sont montrées être plus susceptibles à la variabilité climatique saisonnière, où les modèles climatiques multi-échelles, à travers les téléconnexions et interactions composées, peuvent influencer la dynamique de transmission.
\\
\\
\textbf{Conclusions}: La transmission des maladies transmises par \textit{Aedes} est montrée être susceptible à de multiples facteurs, incluant le changement climatique à long terme et la variabilité saisonnière. Les résultats de cette étude soulignent l'utilisation de comprendre le rôle multiforme des modèles climatiques dans l'émergence des maladies, et leur applicabilité potentielle dans le développement de systèmes d'alerte précoce pour les épidémies de maladies transmises par \textit{Aedes}. La recherche future devrait se concentrer sur l'intégration de ces résultats dans un système de prédiction saisonnière actionnable pour les maladies transmises par \textit{Aedes} utilisant les modèles climatiques comme prédicteurs, qui peut ultimement mieux informer les stratégies de santé publique pour les futures épidémies de dengue.
\\
\\
\textbf{Mots-clés}: Santé Publique, Maladies à Transmission Vectorielle, Épidémiologie, Changement Climatique, Services Climatiques, Sciences Environnementales

\newpage

\subsection*{Русский (Russian)}

\textbf{Предпосылки}: Трансмиссивные заболевания, передаваемые комарами \textit{Aedes}, такие как лихорадка денге, Зика и чикунгунья, представляют значительные вызовы общественному здравоохранению во всем мире в условиях антропогенного изменения климата. Однако, хотя известно, что их передача чувствительна к некоторым климатическим переменным, таким как температура, осадки или влажность, общая роль климатических паттернов в возникновении этих заболеваний не так хорошо понята.
\\
\\
\textbf{Методы}: Используя данные из различных источников, мы исследуем и анализируем реакцию передачи заболеваний, переносимых \textit{Aedes}, на климатические паттерны, чтобы понять их влияние на вспышки заболеваний. Наш анализ состоит из трех различных исследований: 1) временное разложение значений трансмиссивности заболеваний, тем самым направляя чиновников к пониманию поведения вспышек для распределения бюджета и ресурсов; 2) корреляционный анализ между значениями трансмиссивности и различными климатическими паттернами, такими как Южное колебание Эль-Ниньо, чтобы понять влияние естественных климатических паттернов на вспышки, переносимые \textit{Aedes}; и 3) анализ причинности для укрепления результатов, полученных посредством корреляции, выявляя наиболее релевантные предикторы для заболеваний, переносимых \textit{Aedes}.
\\
\\
\textbf{Результаты}: Показано, что долгосрочное антропогенное изменение климата оказывает значительное влияние на экологическую пригодность для заболеваний, переносимых \textit{Aedes}, в тропиках, где Южное колебание Эль-Ниньо и бассейн Индийского океана являются ключевыми климатическими паттернами, обусловливающими возникновение заболеваний. Умеренные регионы показаны более восприимчивыми к сезонной климатической изменчивости, где многомасштабные климатические паттерны через телесвязи и составные взаимодействия могут влиять на динамику передачи.
\\
\\
\textbf{Выводы}: Показано, что передача заболеваний, переносимых \textit{Aedes}, восприимчива к множественным факторам, включая долгосрочное изменение климата и сезонную изменчивость. Результаты этого исследования подчеркивают использование понимания многогранной роли климатических паттернов в возникновении заболеваний и их потенциальной применимости в разработке систем раннего предупреждения для вспышек заболеваний, переносимых \textit{Aedes}. Будущие исследования должны сосредоточиться на интеграции этих результатов в действенную систему сезонного прогнозирования заболеваний, переносимых \textit{Aedes}, используя климатические паттерны как предикторы, что может в конечном итоге лучше информировать стратегии общественного здравоохранения для будущих вспышек лихорадки денге.
\\
\\
\textbf{Ключевые слова}: Общественное здравоохранение, Трансмиссивные заболевания, Эпидемиология, Изменение климата, Климатические службы, Науки об окружающей среде

\newpage

\end{document}
